\documentclass[11pt]{article}

\usepackage{graphics}
\usepackage{url}
\usepackage{verbatim}

%I don't like my paragraphs indented because we have lots of 1-line paragraphs with URLs or code after them
\setlength{\parindent}{0in} 
%I like space between my paragraphs
\setlength{\parskip}{10pt} 

\title{RL-Glue C/C++ Codec 3.0 Manual $Revision$}
\author{Brian Tanner}
\date{$Date$}                                           % Activate to display a given date or no date

\begin{document}
\maketitle
\tableofcontents


\section{Introduction}
This document describes how to use the C/C++ RL-Glue Codec, a software library that provides socket-compatibility with the RL-Glue Reinforcement Learning software library.  

For general information and motivation about the RL-Glue\footnote{\url{http://glue.rl-community.org/}} project, please refer to the documentation provided with that project.

This codec will allow you to create agents, environments, and experiment programs in C and/or C++.

\subsection{Software Requirements}
This project requires that RL-Glue has been installed on your computer.  It has no additional requirements beyond RL-Glue: nothing more exotic than a C compiler, Make, etc.  This codec uses a configure script that was created by GNU Autotools\footnote{\url{http://sources.redhat.com/autobook/}}, so it should compile and run without problems on most *nix platforms (Unix, Linux, Mac OS X, Windows using CYGWIN\footnote{\url{http://www.cygwin.com/}}). 

\subsection{Getting the Codec}
The codec can be downloaded either as a tarball or can be checked out of the subversion repository where it is hosted.

The tarball distribution can be found here:\newline
\url{http://code.google.com/p/rl-glue-ext/downloads/list}


To check the code out of subversion:\newline
\texttt{svn checkout http://rl-glue-ext.googlecode.com/svn/trunk/projects/codecs/C}

\subsection{Installing the Codec}
The codec package was made with autotools, which means that you shouldn't have to do much work to get it installed.  

\subsubsection{Simple Codec Install}
If you are working on your own machine, it is usually easiest to install the headers and libraries into \texttt{/usr/local}, which is the default installation location but requires \textit{sudo} or \textit{root} access.

The steps are:
\begin{verbatim}
	>$ ./configure
	>$ make
	>$ sudo make install
\end{verbatim}

Provided everything goes well, the headers have now been installed to \texttt{/usr/local/include} and the libs to \texttt{/usr/local/lib}.

\subsubsection{Install Codec when RL-Glue is in a custom location}

If \texttt{configure} can't find RL-Glue installed on your machine, it will give you an error like the following:
\begin{verbatim}
checking for rlConnect in -lrlgluenetdev... no
configure: error: RL-Glue library not found.
You must have RL-Glue installed to use this codec. 
	
If you have not downloaded it please see http://glue.rl-community.org/
If you do have it installed in a non-standard location you may need to use the 
--with-rl-glue=/path/to/rlglue 
command line switch to specify where the rl-glue root is located.
\end{verbatim}

If you installed RL-Glue to some place other than \texttt{/usr/local}, say \texttt{/Users/btanner/tmp/rlglue}, you could do:
\begin{verbatim}
	>$ ./configure --with-rl-glue=/Users/btanner/tmp/rlglue/lib
	>$ make
	>$ sudo make install
\end{verbatim}

\subsubsection{Install Codec To Custom Location (without \textit{root} access)}
If you don't have \textit{sudo} or \textit{root} access on the target machine, you can install the codec in your home directory (or other directory you have access to).
If you install to a custom location, you will need set your \texttt{CFLAGS} and \texttt{LDFLAGS} variables appropriately when compiling your agents, environments, and experiments. See Section \ref{sec:custom-flags} for more information.

For example, maybe we want to install the codec to \texttt{/Users/btanner/tmp/rlglue}.  This will \textbf{not} clobber RL-Glue if it is already installed to this location, it will install beside it.  The commands are:
\begin{verbatim}
	>$ ./configure --prefix=/Users/btanner/tmp/rlglue
	>$ make
	>$ make install
\end{verbatim}

Provided everything goes well, the headers and libraries have been respectively installed to\newline
\texttt{/Users/btanner/tmp/rlglue/include} and \texttt{/Users/btanner/tmp/rlglue/lib}.




\section{Agents}
\label{sec:agent}
We have provided a skeleton agent with the codec that is a good starting point for agents that you may write in the future.
It implements all the required functions and provides a good example of how to compile a simple agent.

The pertinent files are:
\begin{verbatim}
	examples/skeleton_agent/SkeletonAgent.c
	examples/skeleton_agent/Makefile
\end{verbatim}

This agent is not particularly interesting, it does not learn anything and randomly chooses integer action $0$ or $1$.  

If RL-Glue and this codec have been installed in the default location, \texttt{/usr/local}, then you can compile and run the agent like:
\begin{verbatim}
	>$ cd examples/skeleton_agent
	>$ make
	>$ ./SkeletonAgent
\end{verbatim}

You will see something like:
\begin{verbatim}
	RL-Glue C Agent Codec Version 1.0-alpha-3, Build 192:208M
		Connecting to host=127.0.0.1 on port=4096...
\end{verbatim}

This means that the SkeletonAgent is running, and trying to connect to the \texttt{rl\_glue} executable server on the local machine through port $4096$! 

See Section \ref{sec:custom-flags} if RL-Glue or this Codec are not installed in default locations.

The Skeleton agent is very simple and well documented, so we won't spend any more time talking about it in these instructions.
Please open it up and take a look.

\textbf{POSSIBLE CONTRIBUTION}: If you take a look at the agent and you think it's not easy to understand, think it could be better documented, 
or just that it should do some fancier things, let us know and we'll be happy to do it!

We will spend a little bit talking about how to compile the agent, because not everyone is comfortable with using a \texttt{Makefile}.  To compile
the agent from the command line, you could do:
\begin{verbatim}
	>$ cc SkeletonAgent.c -lrlutils -lrlagent -o SkeletonAgent
\end{verbatim}

On some platforms, you may need to add \texttt{-lrlgluenetdev}

It might be useful to break this down a little bit:
\begin{description}
\item [cc] The C compiler.  You could also use \texttt{gcc} or \texttt{g++}, etc.
\item [SkeletonAgent.c] Compile the SkeletonAgent.c source file.
\item [-lrlutils] Link to the RLUtils library, which comes with this codec.  This library contains convenience functions for allocating and cleaning up the structure types (Section \ref{sec:structure-types}).  If you 
don't use these convenience functions, you don't need this library.
\item [-lrlagent] Link to the RLAgent library of the codec.  This is where the main agent loop is defined. The main agent loop connects to the \texttt{rl\_glue} executable server and dispatches commands sent by the glue.
\item [-lrlgluenetdev] Link to the RLGlueNetDev library from the RL-Glue project.  This library is automatically linked through \texttt{rlagent} on most platform (except notably Cygwin).  
RLGlueNetDev provides implementations of the low level network code that is used by all three parts of the codec, as well as the \texttt{rl\_glue} executable server.
\end{description}

%Find a place for this
\subsection{Custom Flags for Custom Installs}
\label{sec:custom-flags}
If RL-Glue \textbf{or} this codec have been installed in a custom location (for example: \texttt{/Users/joe/glue}), then you will
need to set the header search path in \texttt{CFLAGS} and the library search path in \texttt{LDFLAGS}.  You can either do this each time you call make, 
or you can export the values as environment variables.  These instructions apply to agents, environments, and experiment programs.

To do it on the command line:
\begin{verbatim}
>$ CFLAGS=-I/Users/joe/glue LDFLAGS=-L/Users/joe/glue make
\end{verbatim}

That might turn out to be quite a hassle while you are developing.  In that case, you can either update the \texttt{Makefile} to include these flags, 
or set an environment variable.  If you are using the bash shell you can \texttt{export} the environment variables:
\begin{verbatim}
>$ export CFLAGS=-I/Users/joe/glue
>$ export LDFLAGS=-L/Users/joe/glue
>$ make
\end{verbatim}

When you open a new terminal window, these values will be lost unless you put the appropriate \texttt{export} lines in your shell startup script.  But that's enough about that, because
we're getting well off topic.

\section{Environments}
We have provided a skeleton environment with the codec that is a good starting point for environments that you may write in the future.
It implements all the required functions and provides a good example of how to compile a simple environment.  This section will follow the same 
pattern as the agent version (Section \ref{sec:agent}).  This section will be less detailed because many ideas are similar or identical.

The pertinent files are:
\begin{verbatim}
	examples/skeleton_environment/SkeletonEnvironment.c
	examples/skeleton_environment/Makefile
\end{verbatim}

This environment is not particularly interesting. It is episodic, with 21 states, labeled $\{0, 1,\ldots,19,20\}$. States $\{0, 20\}$ are terminal and return rewards of $\{-1, +1\}$ respectively.  The other states return reward of $0$.
There are two actions, $\{0, 1\}$.  Action $0$ decrements the state number, and action $1$ increments it. The environment starts in state 10.

If RL-Glue and this codec have been installed in the default location, \texttt{/usr/local}, then you can compile and run the environment like:
\begin{verbatim}
	>$ cd examples/skeleton_environment
	>$ make
	>$ ./SkeletonEnvironment
\end{verbatim}

You will see something like:
\begin{verbatim}
	RL-Glue C Environment Codec Version 1.0-alpha-3, Build 192:208M
		Connecting to host=127.0.0.1 on port=4096...
\end{verbatim}

This means that the SkeletonEnvironment is running, and trying to connect to the \texttt{rl\_glue} executable server on the local machine through port $4096$! 

See Section \ref{sec:custom-flags} if RL-Glue or this Codec are not installed in default locations.

The Skeleton environment is very simple and well documented, so we won't spend any more time talking about it in these instructions.
Please open it up and take a look.

\textbf{POSSIBLE CONTRIBUTION}: If you take a look at the environment and you think it's not easy to understand, think it could be better documented, 
or just that it should do some fancier things, let us know and we'll be happy to do it!

Compiling the environment is almost identical to compiling the skeleton agent, except you need to link to the \texttt{RLEnvironment} library instead of \texttt{RLAgent}.
\begin{verbatim}
	>$ cc SkeletonEnvironment.c -lrlutils -lrlenvironment -o SkeletonEnvironment
\end{verbatim}

On some platforms, you may need to add \texttt{-lrlgluenetdev}

\section{Experiments}
We have provided a skeleton experiment with the codec that is a good starting point for experiment that you may write in the future.
It implements all the required functions and provides a good example of how to compile a simple experiment.  This section will follow the same 
pattern as the agent version (Section \ref{sec:agent}).  This section will be less detailed because many ideas are similar or identical.

The pertinent files are:
\begin{verbatim}
	examples/skeleton_experiment/SkeletonExperiment.c
	examples/skeleton_experiment/Makefile
\end{verbatim}

This experiment runs \texttt{RL\_Episode} a few times, sends some messages to the agent and environment, and then steps through one episode using \texttt{RL\_step}.

If RL-Glue and this codec have been installed in the default location, \texttt{/usr/local}, then you can compile and run the experiment like:
\begin{verbatim}
	>$ cd examples/skeleton_experiment
	>$ make
	>$ ./SkeletonExperiment
\end{verbatim}

You will see something like:
\begin{verbatim}
	RL-Glue C Experiment Codec Version 1.0-alpha-3, Build 192:208M
		Connecting to host=127.0.0.1 on port=4096...
\end{verbatim}

This means that the SkeletonExperiment is running, and trying to connect to the \texttt{rl\_glue} executable server on the local machine through port $4096$! 

See Section \ref{sec:custom-flags} if RL-Glue or this Codec are not installed in default locations.

The Skeleton experiment is very simple and well documented, so we won't spend any more time talking about it in these instructions.
Please open it up and take a look.

\textbf{POSSIBLE CONTRIBUTION}: If you take a look at the experiment and you think it's not easy to understand, think it could be better documented, 
or just that it should do some fancier things, let us know and we'll be happy to do it!

Compiling the experiment is almost identical to compiling the skeleton agent, except you need to link to the \texttt{RLExperiment} library instead of \texttt{RLAgent}.
\begin{verbatim}
	>$ cc SkeletonExperiment.c -lrlutils -lrlexperiment -o SkeletonExperiment
\end{verbatim}

On some platforms, you may need to add \texttt{-lrlgluenetdev}

\subsection{Gotchas!}
\begin{itemize}
	\item If you are calling \texttt{RL\_step}, beware that the last step (when \texttt{terminal==1}), the action will be empty.
\end{itemize}


\section{Putting it all together}
Explain how to run the skeletons together.

\section{Who creates and frees memory?}
- Copy-when-keep





\section{Codec Specification Reference}
This section will explain how the RL-Glue types and functions are defined for this codec.  This isn't meant to be an interesting section of this document, but it will
be handy.

\subsection{Types}
The types used by this codec are the same as the direct-compile RL-Glue library.


\subsection{Simple Types}
The simple types are:

\begin{verbatim}
	typedef double reward_t;
	typedef int terminal_t;
	typedef char* message_t;
	typedef char* task_specification_t;
\end{verbatim}

\def\rat{rl\_abstract\_type\_t}

\subsection{Structure Types}
\label{sec:structure-types}
All of the major structure types (observations, actions, random seed keys, and state keys) are typedef'd to \texttt{\rat}.

\begin{verbatim}
typedef struct
{
	unsigned int numInts;
	unsigned int numDoubles;
	unsigned int numChars;
	int* intArray;
	double* doubleArray;
	char* charArray;
} rl_abstract_type_t;
\end{verbatim}

The specific names and definitions of the structure types are:
\begin{verbatim}
	typedef rl_abstract_type_t observation_t;
	typedef rl_abstract_type_t action_t;
	typedef rl_abstract_type_t random_seed_key_t;
	typedef rl_abstract_type_t state_key_t;
\end{verbatim}

The composite structure types returned by \texttt{env\_step} are:
\begin{verbatim}
	typedef struct{
	  observation_t o;
	  action_t a;
	} observation_action_t;

	typedef struct
	{
	  reward_t r;
	  observation_t o;
	  terminal_t terminal;
	} reward_observation_t;

	typedef struct {
	  reward_t r;
	  observation_t o;
	  action_t a;
	  terminal_t terminal;
	} reward_observation_action_terminal_t;
\end{verbatim}

\subsection{Summary}
The type names are:
\begin{verbatim}
	reward_t
	terminal_t
	message_t
	task_specification_t
	observation_t
	action_t
	observation_action_t
	reward_observation_t
	reward_observation_action_t
\end{verbatim}



\section{Changes and 2.x Backward Compatibility}
\subsection{Types}
All of the types that existed in the 2.x codec (ex: \texttt{Observation} instead of \texttt{observation\_t}) are still supported through a definition file called 
\texttt{legacy\_types.h}.  If you don't want to update your old agents to the new types, you can use the old names by doing the following in your source files:
\begin{verbatim}
	#include<rlglue/legacy_types.h>
\end{verbatim}



\section{Frequently Asked Questions}

\section{Credits and Acknowledgements}
Thanks to everyone for testing the code, and for helping out.

\subsection{Contributing}
If you would like to become a member of this project and contribute updates/changes to the code, please send a message to rl-glue@googlegroups.com.
\end{document}  